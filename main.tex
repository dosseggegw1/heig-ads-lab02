\documentclass[a4paper,11pt]{article}
\usepackage[T1]{fontenc}
\usepackage[utf8]{inputenc}
\usepackage{lmodern}
\usepackage[francais]{babel}
\usepackage{fullpage}

\title{ADS Lab 02 - Aide en ligne Unix et manipulation de fichiers et répertoires}
\date{03 mars 2021}
\author{Gabriel Roch \and Gwendoline Dössegger}
\newcounter{commande}[subsection]
\newcommand{\question}[1]{\addtocounter{commande}{1}\paragraph{Question \arabic{commande}, #1\\}}
\begin{document}

\maketitle
\section{Outil de pagination less}
\question{Avec quelle touche se déplace-t-on au début du fichier?}
  g ou <

\question{Avec quelle touche se déplace-t-on à la fin du fichier?}
  G ou >

\question{Comment peut-on comemncer à chercher toutes les occurences du motif admin ?}
  /admin

\question{Comment se déplace-t-on à l'occurence suivante ?}
  n

\question{Quel est l'effet de taper -i avant une recherche ? Et un -i répété ? Positionnez-vous au début du fichier et chercher le motif bob}
  Avant une recherche : on ignore la case pour une recherche d'un motif ne contenant pas de majuscule
  Avec un -i répété   : désactive l'option


\question{Que se passe-t-il si vous ouvrez}
\begin{verbatim}
  un fichier comprimé, par exemple /usr/share/doc/bash/INTRO.gz ?
\end{verbatim}
  Le fichier est décompressée à la volée et est affiché dans le terminal. 

\begin{verbatim}
  une archive comprimée, par exemple /usr/share/doc/apg/php.tar.gz ?
\end{verbatim}
  Affiche le détail du contenu de l'archive.


\section{Naviguer dans le manuel UNIX}
\question{Quelles sont les neufs sections du manuel Unix ?}
\begin{enumerate}
  \item Commande utilisateur et application
  \item Appel système
  \item Appel des bibliothèques
  \item Fichiers spéciaux
  \item Format des fichiers
  \item Les jeux, économiseurs d'écran et gadgets, etc.
  \item Divers et commandes non standards
  \item Commandes d'administration système 
  \item Sous-programme du noyau
\end{enumerate}
\question{Quelle section du manuel contient les commandes utilisateur telles que cat et ls ?}


\question{Quelle section documente les formats de fichier, tels que les fichiers de configuration ?}
\question{Quelle}


\section{Structure des "MAN PAGES"}
\subsection{Section synopsis}
\subsection{Autres sections}
\subsection{Résoudre des ambiguïtés}
\subsection{La commande ls}
\section{Chemins absolus et relatifs}
\section{Manipulation de fichiers}
\section{Globbing}
\end{document}
